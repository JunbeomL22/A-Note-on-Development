\documentclass[a4paper, 11pt]{article}              % for a regular run         
\RequirePackage[colorlinks,citecolor=blue,urlcolor=blue]{hyperref}     
\DeclareFontShape{OT1}{cmr}{bx}{sc}{<-> Combs10}{}   
%\usepackage{libertine}      
\usepackage[oldstylenums]{kpfonts}     
\usepackage{amsmath, amssymb, amsthm} 
\usepackage{eufrak}
\usepackage{hhline}
\usepackage{kotex}
\usepackage{yfonts}
\usepackage{soul}
\usepackage[titletoc]{appendix}
\usepackage{microtype}
\usepackage{mathtools}
\usepackage[edges]{forest}
\usepackage{cleveref}\crefname{equation}{}{}
\usepackage{enumerate} 
\usepackage{bm} 
\usepackage[utf8]{inputenc} 
\usepackage[T1]{fontenc}
\usepackage[normalem]{ulem}
\usepackage[]{geometry}
\usepackage{authblk}
\usepackage{xcolor}
%\usepackage[euro]{textcomp}
\numberwithin{equation}{section}  
\geometry{textwidth=16.5cm, bottom = 3cm, top = 3cm}  
%\usepackage[doublespacing]{setspace} %onehalfspacing
%\usepackage{lineno}
%\linenumbers
\usepackage{dsfont}
\usepackage{graphicx, epstopdf}
%\epstopdfsetup{outdoor=./} 
\usepackage{subcaption}
\usepackage[]{apacite} 
\usepackage[round, sort, compress, authoryear]{natbib} 
\bibliographystyle{apacite}    
\theoremstyle{plain}
\newcommand{\scrE}{\mathscr{E}}
\DeclareMathOperator*{\dom}{dom} 
\newcommand{\FBA}{\text{FBA}}
\newcommand{\FCA}{\text{FCA}}
\newcommand{\FVA}{\text{FVA}}
\newcommand{\DVA}{\text{DVA}}
\newcommand{\CVA}{\text{CVA}}
\newcommand{\FBAd}{\text{FBA}^\Delta}
\newcommand{\FVAd}{\text{FVA}^\Delta}
\newcommand{\FCAd}{\text{FCA}^\Delta}
\newcommand{\DVAd}{\text{DVA}^\Delta}
\newcommand{\CVAd}{\text{CVA}^\Delta}
\newcommand{\dqdt}{\df \dsQ \otimes \df t-\text{a.s}}
\newcommand{\dqdj}{\df \dsQ \otimes \df \theta-\text{a.s}}
\newcommand{\dpdt}{\df \dsP \otimes \df t-\text{a.s}}
\newcommand{\btau}{\bar{\tau}} 
\newcommand{\jb}{\bar{\theta}}
\newcommand{\gratio}{\frac{\gamma^C}{\gamma^H}}
\newcommand{\tb}{\bar{\tau}}
\newcommand{\gb}{\bar{g}}
\newcommand{\yb}{\bar{y}}
\newcommand{\cadlag}{c\`adl\`ag}
\newcommand{\fb}{\bar{f}}
\newcommand{\calY}{\mathcal{Y}}
\newcommand{\calZ}{\mathcal{Z}}
\newcommand{\Vb}{\bar{V}}
\newcommand{\udld}{\underline{\delta}}
\newcommand{\ovld}{\overline{\delta}}
\newcommand{\rb}{\bar{r}}
\newcommand{\brf}{\bar{f}}
\newcommand{\IDSR}{\int_{\mathds{R}}}
\newcommand{\Gb}{\bar{G}}
\newcommand{\GQ}{\Gamma^{\dsQ}}
\newcommand{\dsone}{\mathds{1}}
\newcommand{\1}{\mathds{1}}
\newcommand{\calF}{\mathcal{F}}
\newcommand{\calT}{\mathcal{T}}
\newcommand{\calC}{\mathcal{C}}
\newcommand{\Bz}{\mathcal{B}_0}
\newcommand{\calI}{\mathcal{I}}
\newcommand{\calB}{\mathcal{B}}
\newcommand{\calA}{\mathcal{A}}
\newcommand{\calD}{\mathcal{D}}
\newcommand{\frD}{\textfrak{D}}
\newcommand{\frA}{\mathfrak{A}} 
\newcommand{\frV}{\mathfrak{V}}
\newcommand{\frC}{\mathfrak{C}}
\newcommand{\frM}{\mathfrak{M}}
\newcommand{\calV}{\mathcal{V}}
\newcommand{\calE}{\mathcal{E}}
\newcommand{\calH}{\mathcal{H}}
\newcommand{\calG}{\mathcal{G}}
\newcommand{\calP}{\mathcal{P}}
\newcommand{\calO}{\mathcal{O}}
\newcommand{\calR}{\mathcal{R}}
\newcommand{\calL}{\mathcal{L}}
\newcommand{\Nt}{\tilde{N}}
\newcommand{\Bt}{\tilde{B}}
\newcommand{\St}{\tilde{S}}
\newcommand{\Sc}{\check{S}}
\newcommand{\Xc}{\check{X}}
\newcommand{\cc}{\check{c}}
\newcommand{\Vc}{\check{V}}
\newcommand{\pic}{\check{\pi}}
\newcommand{\nt}{\tilde{\nu}}
\newcommand{\Lt}{\tilde{L}}
\newcommand{\ct}{\tilde{C}_{t}}
\newcommand{\yt}{\tilde{y}}
\newcommand{\ft}{\tilde{f}}
\newcommand{\fh}{\hat{f}}
\newcommand{\Zh}{\hat{Z}}
\newcommand{\Yh}{\hat{Y}}
\newcommand{\Xh}{\hat{X}}
\newcommand{\xih}{\hat{\xi}}
\newcommand{\Ut}{\tilde{U}}
\newcommand{\Utt}{\tilde{\tilde{U}}}
\newcommand{\Vt}{\tilde{X}_t}
\newcommand{\Vs}{\tilde{X}_s}
\newcommand{\pt}{\tilde{P}}
\newcommand{\pc}{\check{p}}
\newcommand{\yh}{\hat{y}}
\newcommand{\zh}{\hat{z}}
\newcommand{\Gh}{\hat{G}}
\newcommand{\kh}{\hat{k}}
\newcommand{\mh}{\hat{m}}
\newcommand{\hh}{\hat{h}}
\newcommand{\Uh}{\hat{U}}
\newcommand{\jh}{\hat{j}}
\newcommand{\pit}{\tilde{\pi}}
\newcommand{\eps}{\varepsilon}
\newcommand{\et}{\tilde{e}}
\newcommand{\dsI}{\mathbb{I}}  
\newcommand{\dsE}{\mathbb{E}}
\newcommand{\dsF}{\mathbb{F}}
\newcommand{\dsH}{\mathbb{H}}
\newcommand{\dsQ}{\mathbb{Q}} 
\newcommand{\dsP}{\mathbb{P}}
\newcommand{\dsA}{\mathbb{A}}
\newcommand{\dsL}{\mathbb{L}}
\newcommand{\dsR}{\mathbb{R}}
\newcommand{\dsS}{\mathbb{S}}
\newcommand{\dsG}{\mathbb{G}}
\newcommand{\dsD}{\mathbb{D}}
\newcommand{\tV}{\tilde{X}}
\newcommand{\tP}{\tilde{P}}
\newcommand{\Jt}{\tilde{\Theta}}
\newcommand{\Jh}{\hat{\Theta}}
\newcommand{\INT}{\int_{t+}^{\btau}}
\newcommand{\SUM}{\displaystyle\sum}
\newcommand{\ccs}{U_C\big(\1_{T < \tau}\upsilon + \1_{T\geq \tau}(v_\tau - \Theta_\tau)-B_{\btau}^C p}
\newcommand{\IT}{^i_t}
\newcommand{\lh}{\hat{\lambda}}
\newcommand{\Hh}{\hat{H}}
\newcommand{\hb}{\bar{h}}
\newcommand{\ph}{\hat{p}}
\newcommand{\pih}{\hat{\pi}}
\newcommand{\Gmh}{\hat{\Gamma}}
\newcommand{\gmh}{\hat{\gamma}}
\newcommand{\Bb}{\bar{B}}
\newcommand{\gh}{\hat{g}}
\newcommand{\ch}{\hat{c}}
\newcommand{\bh}{\hat{\beta}}
\newcommand{\uh}{\hat{u}}
\newcommand{\xf}{\mathbf{x}}
\newcommand{\tf}{\mathbf{t}}
\newcommand{\yf}{\mathbf{y}}
\newcommand{\zf}{\mathbf{z}}
\newcommand{\lf}{\mathbf{\lambda}}
\newcommand{\IS}{^i_s}
\newcommand{\RI}{\rho^I}
\newcommand{\Ub}{\bar{U}}
\newcommand{\RC}{\rho^C}
\newcommand{\agrc}{\textit{agreement-cost}}
\DeclareMathOperator*{\argmin}{arg\,min}
\DeclareMathOperator*{\argmax}{arg\,max}
\DeclareMathOperator*{\esssup}{ess\,sup}
%\DeclareMathOperator*{\exp}{exp}
\DeclareMathOperator*{\essinf}{ess\,inf}
%\usepackage[linedheaders,parts,pdfspacing]{classicthesis} % ,manychapters
%\usepackage[bitstream-charter]{mathdesign}
\usepackage{bbding}
\usepackage{thmtools}
    
\theoremstyle{plain}
\newtheorem{thm}{Theorem}[section] % reset theorem numbering for each chapter
\newtheorem{corollary}[thm]{Corollary} 
\newtheorem{theorem}[thm]{Theorem} 
\newtheorem{lemma}[thm]{Lemma}
\newtheorem{proposition}[thm]{Proposition}
\newtheorem{assumption}[thm]{Assumption}
\newtheorem{definition}[thm]{Definition} 

\theoremstyle{definition}
\newtheorem{remark}[thm]{Remark} % definition numbers are dependent on theorem numbers
\newtheorem{example}[thm]{Example}
 \iffalse 
\declaretheoremstyle[
  style=definition,
  qed=\openbox,
]{example}
\declaretheorem[
  name=Example,
  style=example,
  numberlike=thm,
  ]{exmp}
\fi
\renewcommand\qedsymbol{\RectangleBold}  
\iffalse
\newtheorem{assumption}{Assumption}
\newtheorem{corollary}{Corollary}
\newtheorem{definition}{Definition}
\newtheorem{lemma}{Lemma}
\newtheorem{theorem}{Theorem}
\theoremstyle{plain}
\newtheorem{example}{Example}
\newtheorem{remark}{Remark}
\fi
\newcommand{\CT}{^{x+p, u, c}}
\newcommand{\gt}{\tilde{g}}
\newcommand{\uf}{\mathbf{u}}
\newcommand{\Ah}{\hat{A}}
\newcommand*\df{\mathop{}\!\mathrm{d}}
\newcommand*\Df[1]{\mathop{}\!\mathrm{d^#1}}
\newcommand{\comJ}[1]{(\textcolor[rgb]{1,0., 0.0}{*#1})}
\newcommand{\CJ}[1]{\textcolor[rgb]{1.0,0., 0.0}{#1}}
\newcommand{\Rom}[1]{\uppercase\expandafter{\romannumeral #1\relax}}
\newcommand{\rom}[1]{\lowercase\expandafter{\romannumeral #1\relax}}
\newcommand{\wt}[1]{\widetilde{#1}}
% in the document body:  
\newcommand{\RN}[1]{%  
  \textup{\uppercase\expandafter{\romannumeral#1}}% 
}
\newcommand{\citelist}{}

\newcounter{current-cite}

\newcounter{currentcitetotal}
\newcommand{\mycite}[1]{
  \setcounter{currentcitetotal}{0}
  \renewcommand{\do}[1]{\addtocounter{currentcitetotal}{1}}
  \docsvlist{#1}
  \renewcommand{\do}[1]{%
  \addtocounter{currentcite}{1}%
  \ifinlist{##1}{\citelist}
    {\citet{##1}}
    {\citet*{##1}\listadd{\citelist}{##1}}%
  \ifnumcomp{\value{currentcitetotal}}{>}{\value{currentcite}}
    {, }
    {}%
  }
  \docsvlist{#1}
}
\renewenvironment{abstract}{\small
  \begin{center}
  \bfseries \abstractname\vspace{-.5em}\vspace{0pt}
  \end{center}
  \list{}{
    \setlength{\leftmargin}{0.7cm}% 
    \setlength{\rightmargin}{\leftmargin}%
  }%
  \item\relax}   
 {\endlist} 
 \DeclareMathOperator{\sgn}{sgn} 
 \DeclareMathOperator{\diag}{diag} 
 \DeclareMathOperator{\row}{row} 
 \DeclareMathOperator{\col}{col}
    
 \title{A Note on Development  \author{이준범 \thanks{ OTC Trading, Yuanta Securities Korea, 04538
       Seoul, Korea. \\Tel: +82-2-3770-5993. 
       Email: \href{junbeom.lee@yuantakorea.com}{junbeom.lee@yuantakorea.com}.
     %
     %
     }} 
   }
 %\affil{Department of Mathematics\\ National University of Singapore}

\iffalse 
\author{이준범\footnote{OTC Trading, Yuanta Securities
    Korea, 04538 Seoul, Korea
    (\href{junbeom.lee@yuantakorea.com}{junbeom.lee@yuantakorea.com}).
    }} 
\fi    
\date{}
\providecommand{\keywords}[1]{\textbf{\textit{Key-words: }} #1}
\providecommand{\AMS}[1]{\textbf{\textit{AMS subject classifications: }} #1}
\begin{document}
\maketitle

\iffalse
{
  \hypersetup{linkcolor = blue} 
  \tableofcontents
}
\fi
\section{Payoff Decomposition}
\label{sec:back}

ELS Payoff, denoted by $X$ is like:
\begin{align}
  X =& \SUM_{i=1}^N \Big(\prod_{j <i}\1_{S_{T_j} < \calR_j}\Big)\1_{ S_{T_i}
      \geq  \calR_i}C_i
      + \SUM_{i=1}^L\Big(\prod_{j <i}\1_{S_{T_j} < \calR_j}\Big)
      \1_{ S^*_{(T_{i-1} , T_i]} \geq  \calL_i}C^L_i \nonumber \\
     &+\Big(\prod_{j <N}\1_{S_{T_j} < \calR_j}\Big)\big(\1_{S^*_T <
       \calB}\phi(S_T)+\1_{S^*_T \geq \calB}C^B\big)  \nonumber \\
  =& \SUM_{i=1}^N \Big(\prod_{j <i}\1_{S_{T_j} < \calR_j}\Big)\1_{ S_{T_i}
      \geq  \calR_i}C_i
      + \SUM_{i=1}^L\Big(\prod_{j <i}\1_{S_{T_j} < \calR_j}\Big)
      \1_{ S^*_{(T_{i-1} , T_i]} \geq  \calL_i}C^L_i \nonumber \\
     &+\Big(\prod_{j <N}\1_{S_{T_j} < \calR_j}\Big)\big(\phi(S_T)+\1_{S^*_T \geq
       \calB}(C^B-\phi(S_T))\big)
       \nonumber \\
    =& \SUM_{i=1}^{N-1} \Big(\prod_{j <i}\1_{S_{T_j} < \calR_j}\Big)\1_{ S_{T_i}
       \geq  \calR_i}C_i+\Big(\prod_{j <N-1}\1_{S_{T_j} < \calR_j}\Big)
       \big(\1_{S_{T_N} \geq \calR_N}C_N + \1_{S_{T_N} < \calR_N}\phi(S_T)\big)
        \nonumber \\
     &+ \SUM_{i=1}^L\Big(\prod_{j <i}\1_{S_{T_j} < \calR_j}\Big)
       \1_{ S^*_{(T_{i-1} , T_i]} \geq  \calL_i}C^L_i \nonumber\\
     &+\Big(\prod_{j <N}\1_{S_{T_j} < \calR_j}\Big)\1_{S^*_T \geq \calB}(C^B-\phi(S_T))
\end{align}
Hence there should be $L+2$ grids: one for ranges, $L$ for the lizard barriers, one
for the maturity barrier. 

\section{OIS Schedule}
\label{sec:ois}
Let us denote the calculation period pair of fixing, value start,
value end, payment dates by:
\begin{align}
  I_i = \{\phi_i, \sigma_i, \epsilon_i\}.
\end{align}
To explain the notation, we usually have $\epsilon_i - \sigma_i = \text{Day(1)}$
in overnight index cases. These periods comprise the swap intervals
$\{\{S_i, E_i, \Pi_i\}\}_i$ such that:
\begin{align}
  (S_i, E_i] \coloneqq (\sigma_{M_i}, \epsilon_{N_i}] = \cup(\sigma_k, \epsilon_k],
\end{align}
and $\Pi_i$ represents the regarding payment date. 
Then, for a given evaluation date $t \geq 0$, find the earliest payment date and
its valu end date:
\begin{align}
  t \mapsto \Pi_{i(t)} \mapsto E_{i(t)}.
\end{align}
The next step is to find the latest end date of the past data earlier than
$E_{i(t)}$:
\begin{align}
  E_{i(t)} \geq \epsilon^*_{i(t)}
  \coloneqq \{\epsilon_i \mid (\phi_i, \sigma_i, \epsilon_i) \text{ of the past data}\}.
\end{align}
In addition, find the earlist value start date (of the fixing data) and fixing
date:
\begin{align}
  S_{i(t)} \mapsto (\phi^*_{i(t)}, \sigma_{i(t)}^*)
\end{align}
Then, the calculation starts from:
\begin{align}
  \sigma_{i(t)}^* \wedge S_{i(t)}.
\end{align}
For the forcasting, we obtain:
\begin{align}
  [1+C(0, 0, E_{i(t)}) (E_{i(t)} -\epsilon^*_{i(t)})]
\end{align}
For any case that the next value start date $S_{i(t)+1}$ needs a past data, we
calculate:
\begin{align}
  C(t, S_i \vee 0, E_i) (E_{i} -S_{i})
\end{align}

\section{Local Volatility}

\subsection{No Dividends}
Note that (S)SVI models represents the total variance with respect to:
\begin{align}
y = \ln\big(\frac{K}{F}\big).
\end{align}
In other words:
\begin{align}
  w(y) = \sigma^2(T, Fe^k)T. 
\end{align}
Recall the Dupire formula:
\begin{align}
  v_L = \frac{\partial_T w}{1-\frac{y}{w}\partial_y w
  +\frac{1}{4}\Big(-\frac{1}{4} - \frac{1}{w} +
  \frac{y^2}{w^2}\Big)(\partial_y w)^2 + \frac{1}{2}\partial^2_y w}. \label{dupire}
\end{align}
Hence, recall that $\sigma_{BS}(T, Fe^y) = \sqrt{v_L(y)}$, in other words:
\begin{align}
 \sigma_{BS}(T, K) = \sqrt{v_L\bigg[\ln\big(\frac{K}{F}\big)\bigg]}.
\end{align}
For any $T <  T_1$ and $T >  T_N$, the possible extrapolation may be:
\begin{align}
  \sigma_{BS}\bigg(T_1, ~K\frac{F_1}{F}\bigg) ~\text{ and }~ \sigma_{BS}\bigg(T_N, ~K\frac{F_N}{F}\bigg)
\end{align}

\subsection{Discrete Dividends}
This recipe is borrowed from the section 2.3.1.2 in 
\cite{bergomi2015stochastic}.
\begin{align}
  \delta S(t)\coloneqq& \SUM_{t_i <t}q_i\frac{t-t_i}{t}B_{t_i}^{-1}\nonumber\\
  \delta K(t)\coloneqq& \SUM_{t_i <t}q_i\frac{t_i}{t}B_{t_i}^{-1}B_t \nonumber\\
  y\coloneqq& \ln\bigg(\frac{K+\Delta K(t)}{(S_0 - \delta S(t))B_t}\bigg)\nonumber\\
  w(t, y)\coloneqq& \sigma_{L}^2(t, K)t.\nonumber 
\end{align}
Then, \cref{dupire} implies directly, but:
\begin{align}
  \sigma_{BS}(t, K) = \sqrt{v_L\bigg[\ln\bigg(\frac{K+\Delta K(t)}{(S_0 - \delta S(t))B_t}\bigg)\bigg]}
\end{align}
Moreover, the extrapolation accross time is:
\begin{align}
  \sigma_{BS}\bigg(t_1, ~e^{-y+y_1} \bigg) ~\text{ and }~ \sigma_{BS}\bigg(t_N, ~e^{-y+y_N}\bigg)
\end{align}
\subsection{Referencing the Intial Price}
Recall that the SDE of local volatility model follows:
\begin{align}
  \df S_t = r_tS_t\df t + \sigma_{BS}(t, S_t)S_t\df W_t - \df \frD_t
\end{align}
Let $X_t = S_t / S^*$ where $S^*$ represents the reference price on the
effective date. Then, we have:
\begin{align}
  \df X_t =& r_tX_t\df t + \sigma_{BS}(t, X_tS^*)X_t\df W_t - \frac{1}{S^*}\df
             \frD_t \nonumber \\
  =& r_tX_t\df t + \sigma_{BS}(t, X_tS^*)X_t\df W_t - \SUM \frac{q_i}{S^*}\bm{\delta}_{t_i}(\df t)
\end{align}
\section{Log BS with Discrete Dividends}
Let $(S_t)_{t \geq 0}$ follow:
\begin{align}
  \df S_t =& r_tS_t \df t +  v_L\bigg(t,
             \ln\big(\frac{S_t}{F^t_0}\big)\bigg)^{\frac{1}{2}} S_t \df W_t
             - \df \frD_t \nonumber\\
  \frD_t =& \SUM_{i \leq N}q_i \1_{T_i \leq t}. \nonumber
\end{align}
We formally have:
\begin{align}
  \df \frD_t = \SUM_{i \leq N} q_i \bm{\delta}_{T_i}(\df t).
\end{align}
Let $X = \ln (S / S^*)$. Then:
\begin{align}
  \df X_t =  \big(r_t - \frac{\sigma_t^2}{2}\big) \df t
  +v_L\bigg(t, X_t+\ln\big(\frac{S^*}{F^t_0}\big)\bigg)^{\frac{1}{2}} \df W_t
  +\bm{\delta}_{T_i}(\df t)\big(\ln(e^{X_{t-}}-\frac{q_i}{S^*}) - \ln(e^{X_{t-}})\big) 
\end{align}
Therefore:
\begin{align}
  X_{t+\Delta t} - X_t \approx&
   \big(r_t -
  \frac{\sigma_t^2}{2}\big) \Delta t
  +\sigma_t \Delta W_t
  + \ln\Big(\frac{e^{X_t} - \SUM \frac{q_i}{S^*} \1_{t < T_i \leq t + \Delta
                                t}}{e^{X_t}}\Big) \nonumber\\
  \sigma_t = &v_L\bigg(t, X_t+\ln\big(\frac{S^*}{F^t_0}\big)\bigg)^{\frac{1}{2}} \nonumber
\end{align}
\bibliography{dev.note}
\end{document} 
%%% Local Variables:
%%% mode: latex
%%% TeX-master: t
%%% End:

 